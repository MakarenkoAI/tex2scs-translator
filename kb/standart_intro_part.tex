\scnsectionheader{Стандарт OSTIS}

\scnidtf{Документация \textit{Технологии OSTIS}}
\scnidtf{Документация Открытой \textit{технологии} онтологического проектирования, производства и эксплуатации семантически совместимых гибридных \textit{интеллектуальных компьютерных систем}}
\scnidtf{Описание \textit{Технологии OSTIS} (Open Semantic Technology for Intelligent Systems), представленное в виде раздела \textit{базы знаний ostis-системы} (системы, построенной по \textit{Технологии OSTIS}) на внутреннем языке \textit{ostis-систем} и обладающее достаточной полнотой для использования этой \textit{технологии} разработчиками \textit{интеллектуальных компьютерных систем}}
\scnidtf{Полное описание текущего состояния \textit{Технологии OSTIS}, представленное в виде семейства разделов \textit{базы знаний}, построенной по \textit{Технологии OSTIS}}
\scnidtf{Семейство разделов \textit{базы знаний} \scnbigspace \textit{Метасистемы IMS.ostis}, которое предназначено для комплексной поддержки онтологического проектирования семантически совместимых \textit{гибридных интеллектуальных компьютерных систем}}
\scniselement{семейство разделов базы знаний}
    \scnaddlevel{1}
    \scnidtf{семейство разделов внутреннего представления \textit{базы знаний ostis-системы} -- \textit{интеллектуальной компьютерной системы}, построенной по \textit{Технологии OSTIS}}
    \scnaddlevel{-1}

\scntext{эпиграф}{From data science to knowledge science}
\scntext{аннотация}{В настоящее время информатика преодолевает важнейший этап своего развития -- переход от информатики данных (data science) к информатике знаний (knowledge science), где акцентируется внимание на \uline{семантических} аспектах представления и обработки \textit{знаний}.\\
Без фундаментального анализа такого перехода невозможно решить многие проблемы, связанные с управлением \textit{знаниями}, экономикой \textit{знаний}, с \textit{семантической совместимостью интеллектуальных компьютерных систем}.\\
Основной особенностью \textit{Технологии OSTIS} является ориентация на использование компьютеров нового поколения, специально предназначенных для  реализации семантически совместимых гибридных \textit{интеллектуальных компьютерных систем}. Предлагаемый \textit{Стандарт OSTIS} оформлен в виде \textit{семейства разделов базы знаний} специальной интеллектуальной компьютерной \textit{Метасистемы IMS.ostis} (Intelligent MetaSystem for ostis-systems), которая построена по \textit{Технологии OSTIS} и представляет собой постоянно совершенствуемый интеллектуальный \textit{портал научно-технических знаний}, который поддерживает перманентную эволюцию \textit{Стандарта OSTIS}, а также разработку различных \textit{ostis-систем} (интеллектуальных компьютерных систем, построенных по \textit{Технологии OSTIS}).}
\scnidtf{Процесс перманентной эволюции \textit{Стандарта OSTIS}, совмещенного (интегрированного) с комплексными учебно-методическим обеспечением подготовки специалистов в области \textit{Искуссственного интеллекта} и представленного в виде специального раздела \textit{базы знаний}}
\newpage
\scnnote{Подчеркнем, что \textit{Стандарт OSTIS} -- это не описание некоторого состояния \textit{Технологии OSTIS}, а \uline{динамическая} информационная модель процесса эволюции этой \textit{технологии}}
\scnidtf{Стандарт Технологии OSTIS}
\scnidtf{Документация \textit{Технологии OSTIS}, полностью отражающая \uline{текущее} состояние \textit{Технологии OSTIS} и представленная соответствующим \textit{разделом базы знаний} специальной \textit{ostis-системы}, которая ориентирована на поддержку проектирования, производства, эксплуатации и эволюции (реинжиниринга) \textit{ostis-систем}, а также на поддержку эволюции самой \textit{Технологии OSTIS} и которая названа нами \textit{Метасистемой IMS.ostis}}
\scnidtf{Максимальный раздел \textit{Стандарта OSTIS}, т.е. раздел, в состав которого входят все остальные \textit{разделы} (подразделы) \textit{Стандарта OSTIS}}
\scnidtf{Раздел \textit{базы знаний}, текущее состояние которого отражает текущее состояние (текущую версию) перманентно эволюционируемого Стандарта Комплексной \textit{Технологии OSTIS}}
\scnidtf{Представленное в форме раздела \textit{базы знаний} специальной \textit{ostis-системы} (\textit{Метасистемы IMS.ostis}) полное описание (спецификация, документация) текущего состояния \textit{Технологии OSTIS}}
\scnidtf{Мы рассматриваем \textit{Стандарт OSTIS} (Документацию Стандарта \textit{Технологии OSTIS}) как продукт \textit{научно-технической деятельности}, к которому предъявляются \uline{высокие требования} по полноте, согласованности, непротиворечивости, практической значимости разрабатываемой документации, описывающей текущее состояние \textit{Технологии OSTIS}}

\scnheader{Стандарт OSTIS}
\scnrelto{формальная спецификация}{Технология OSTIS}
\scnaddlevel{1}
\scnidtf{Перманентно развиваемый в рамках открытого проекта комплекс моделей, методов и средств, ориентированных на онтологическое проектирование, производство, экплуатацию и реинжиниринг семантически совместимых гибридных интеллектуальных компьютерных систем, способных самостоятельно взаимодействовать друг с другом}
\scnidtf{Технология разработки семантически совместимых и самостоятельно взаимодействующих интеллектуальных компьютерных систем}
\scnaddlevel{-1}

\bigskip

\scnheader{формальная спецификация*}
\scnsuperset{sc-модель*}
\scnaddlevel{1}
\scnidtf{быть формальной спецификацией (формальной моделью, формальным описанием) заданного объекта, представленной на внутреннем смысловом языке интеллектуальных компьютерных систем (в \mbox{SC-коде})}
\scnaddlevel{-1}

\scnheader{ostis-система}
\scnaddlevel{1}
\scntext{следовательно}{Если \textit{интеллектуальные компьютерные системы} не будут обладать указанными выше способностями, то ни о каких smart-предприятиях, smart-учреждениях, smart-городах, ни о каком smart-обществе и речи быть не может, т.к. все обстоятельства их деятельности заранее на этапе проектирования предусмотреть принципиально невозможно. Это означает, что \textit{интеллектуальные компьютерные системы} должны научиться самостоятельно "отрабатывать" все заранее непредусмотренные обстоятельства.}
\scnaddlevel{-1}
