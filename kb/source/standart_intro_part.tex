\scnsectionheader{Стандарт OSTIS}

\scnidtf{Документация \textit{Технологии OSTIS}}
\scnidtf{Документация Открытой \textit{технологии} онтологического проектирования, производства и эксплуатации семантически совместимых гибридных \textit{интеллектуальных компьютерных систем}}
\scnidtf{Описание \textit{Технологии OSTIS} (Open Semantic Technology for Intelligent Systems), представленное в виде раздела \textit{базы знаний ostis-системы} (системы, построенной по \textit{Технологии OSTIS}) на внутреннем языке \textit{ostis-систем} и обладающее достаточной полнотой для использования этой \textit{технологии} разработчиками \textit{интеллектуальных компьютерных систем}}
\scnidtf{Полное описание текущего состояния \textit{Технологии OSTIS}, представленное в виде семейства разделов \textit{базы знаний}, построенной по \textit{Технологии OSTIS}}
\scnidtf{Семейство разделов \textit{базы знаний} \scnbigspace \textit{Метасистемы IMS.ostis}, которое предназначено для комплексной поддержки онтологического проектирования семантически совместимых \textit{гибридных интеллектуальных компьютерных систем}}
\scniselement{семейство разделов базы знаний}
\begin{scnindent}
\scnidtf{семейство разделов внутреннего представления \textit{базы знаний ostis-системы} -- \textit{интеллектуальной компьютерной системы}, построенной по \textit{Технологии OSTIS}}
\end{scnindent}
\scnrelfrom{финансовая поддержка}{Intelligent Semantic Systems Ltd.}

\scntext{эпиграф}{From data science to knowledge science}
\scntext{аннотация}{В настоящее время информатика преодолевает важнейший этап своего развития -- переход от информатики данных (data science) к информатике знаний (knowledge science), где акцентируется внимание на \uline{семантических} аспектах представления и обработки \textit{знаний}.\\
Без фундаментального анализа такого перехода невозможно решить многие проблемы, связанные с управлением \textit{знаниями}, экономикой \textit{знаний}, с \textit{семантической совместимостью интеллектуальных компьютерных систем}.\\
Основной особенностью \textit{Технологии OSTIS} является ориентация на использование компьютеров нового поколения, специально предназначенных для  реализации семантически совместимых гибридных \textit{интеллектуальных компьютерных систем}. Предлагаемый \textit{Стандарт OSTIS} оформлен в виде \textit{семейства разделов базы знаний} специальной интеллектуальной компьютерной \textit{Метасистемы IMS.ostis} (Intelligent MetaSystem for ostis-systems), которая построена по \textit{Технологии OSTIS} и представляет собой постоянно совершенствуемый интеллектуальный \textit{портал научно-технических знаний}, который поддерживает перманентную эволюцию \textit{Стандарта OSTIS}, а также разработку различных \textit{ostis-систем} (интеллектуальных компьютерных систем, построенных по \textit{Технологии OSTIS}).}
\scnidtf{Процесс перманентной эволюции \textit{Стандарта OSTIS}, совмещенного (интегрированного) с комплексными учебно-методическим обеспечением подготовки специалистов в области \textit{Искуссственного интеллекта} и представленного в виде специального раздела \textit{базы знаний}}
\newpage
\scnnote{Подчеркнем, что \textit{Стандарт OSTIS} -- это не описание некоторого состояния \textit{Технологии OSTIS}, а \uline{динамическая} информационная модель процесса эволюции этой \textit{технологии}}
\scnidtf{Стандарт Технологии OSTIS}
\scnidtf{Документация \textit{Технологии OSTIS}, полностью отражающая \uline{текущее} состояние \textit{Технологии OSTIS} и представленная соответствующим \textit{разделом базы знаний} специальной \textit{ostis-системы}, которая ориентирована на поддержку проектирования, производства, эксплуатации и эволюции (реинжиниринга) \textit{ostis-систем}, а также на поддержку эволюции самой \textit{Технологии OSTIS} и которая названа нами \textit{Метасистемой IMS.ostis}}
\scnidtf{Максимальный раздел \textit{Стандарта OSTIS}, т.е. раздел, в состав которого входят все остальные \textit{разделы} (подразделы) \textit{Стандарта OSTIS}}
\scnidtf{Раздел \textit{базы знаний}, текущее состояние которого отражает текущее состояние (текущую версию) перманентно эволюционируемого Стандарта Комплексной \textit{Технологии OSTIS}}
\scnidtf{Представленное в форме раздела \textit{базы знаний} специальной \textit{ostis-системы} (\textit{Метасистемы IMS.ostis}) полное описание (спецификация, документация) текущего состояния \textit{Технологии OSTIS}}
\scnidtf{Мы рассматриваем \textit{Стандарт OSTIS} (Документацию Стандарта \textit{Технологии OSTIS}) как продукт \textit{научно-технической деятельности}, к которому предъявляются \uline{высокие требования} по полноте, согласованности, непротиворечивости, практической значимости разрабатываемой документации, описывающей текущее состояние \textit{Технологии OSTIS}}

\scnheader{официальная версия Стандарта OSTIS}
\scnidtf{официально издаваемая (публикуемая в бумажном и/или электронном виде) \textit{версия Стандарта OSTIS}}
\scnhaselement{Стандарт OSTIS-2021}
\begin{scnindent}
\scnidtf{Официальная \textit{версия Стандарта OSTIS}, издаваемая непостредственно перед началом \textit{Конференции OSTIS-2021}}
\end{scnindent}

\scnheader{Стандарт OSTIS}
\scnrelto{формальная спецификация}{Технология OSTIS}
\begin{scnindent}
\scnidtf{Перманентно развиваемый в рамках открытого проекта комплекс моделей, методов и средств, ориентированных на онтологическое проектирование, производство, экплуатацию и реинжиниринг семантически совместимых гибридных интеллектуальных компьютерных систем, способных самостоятельно взаимодействовать друг с другом}
\scnidtf{Технология разработки семантически совместимых и самостоятельно взаимодействующих интеллектуальных компьютерных систем}
\scnexplanation{\textit{Технология OSTIS} -- это технология принципиально нового уровня, это обусловлено:
    \begin{scnitemize}
        \item высоким качеством интеллектуальных компьютерных систем (ostis-систем), разрабатываемых на ее основе -- их семантической совместимостью, способностью к самостоятельному взаимодействию, способностью адаптироваться к пользователям и способностью адаптировать (обучать) самих пользователей более эффективному взаимодействию с интеллектальными компьютерными системами;
        \item высоким качеством самой \textit{технологии} -- возможностью интегрировать самые различные \textit{виды знаний} и самые различные \textit{модели решения задач}, неразрывной связью процесса разработки интеллектуальных компьютерных систем и процесса повышения квалификации разработчиков.
    \end{scnitemize}}
\end{scnindent}
\bigskip

\scnheader{формальная спецификация*}
\scnidtf{Бинарное ориентированное \textit{отношение}, каждая \textit{пара} которого связывает
    \begin{scnitemize}
        \item некий формальный текст, являющийся формальной спецификацией (формальной моделью, формальным описанием) с
        \item некой сущностью, которая является \uline{объектом} указанной спецификации (моделирования, описания)
    \end{scnitemize}}

\scnsuperset{sc-модель*}
\begin{scnindent}
\scnidtf{быть формальной спецификацией (формальной моделью, формальным описанием) заданного объекта, представленной на внутреннем смысловом языке интеллектуальных компьютерных систем (в \mbox{SC-коде})}
\end{scnindent}

\scnheader{ostis-система}
\scnidtf{\textit{интеллектуальная компьютерная система}, построенная по \textit{Технологии OSTIS} (по стандартам \textit{Технологии OSTIS}), что обеспечивает:
    \begin{scnitemize}
        \item \textit{семантическую совместимость} (взаимопонимание) всех этих систем между собой;
        \item их \textit{способность к самостоятельному взаимодействию}, к координации своей деятельности при возникновении заранее непредсказуемых (нештатных) ситуаций.
    \end{scnitemize}}
\begin{scnindent}
\scntext{следовательно}{Если \textit{интеллектуальные компьютерные системы} не будут обладать указанными выше способностями, то ни о каких smart-предприятиях, smart-учреждениях, smart-городах, ни о каком smart-обществе и речи быть не может, т.к. все обстоятельства их деятельности заранее на этапе проектирования предусмотреть принципиально невозможно. Это означает, что \textit{интеллектуальные компьютерные системы} должны научиться самостоятельно "отрабатывать" все заранее непредусмотренные обстоятельства.}
\end{scnindent}

\scnrelfromlist{направления эволюции}{\scnfileitem{Конструктивное пополнение \textit{Стандарта OSTIS} на уровне исходного текста базы знаний \textit{Метасистемы IMS.ostis} (продумать при этом и технически поддерживать методики согласования авторских позиций
    \begin{scnitemize}
        \item При четкой фиксации авторских прав и разных точек зрения
        \item Через регулярные коллективные обсуждения -- семинары.
    \end{scnitemize}};
\scnfileitem{Совершенствование формы и \uline{стиля} представления материала};
\scnfileitem{Совершенствование систематизации (структуризации, стратификации) материала, четкая фиксация наследования свойств};
\scnfileitem{Четкий формальный сравнительный анализ со всеми близкими подходами (Semantic Web, графовые базы данных и др.)};\scnfileitem{"Наращивание" \textit{Стандарта OSTIS} соответствующей учебно-методической информацией};
\scnfileitem{Интеграция в состав \textit{Стандарта OSTIS} \uline{всех}(!) развиваемых в настоящее время в области \textit{Искусственного интеллекта} моделей представления знаний, моделей решения задач, моделей интерфейсов при обеспечении их \uline{семантической совместимости}. Весь современный арсенал результатов работ в области \textit{Искусственного интеллекта} в перспективе должен быть интегрирован в \textit{Стандарт OSTIS}, но не как мозаика ("зоопарк") имеющихся трудносовместимых моделей, методов и средств построения \textit{интеллектуальных компьютерных систем}, имеющихся сервисов и информационных ресурсов};
\scnfileitem{Загрузка исходного текста \textit{Стандарта OSTIS} в состав \textit{Базы знаний IMS.ostis} и организация дальнейшей работы по развитию \textit{Стандарта OSTIS} как коллективной разработки \textit{Базы знаний IMS.ostis}};
\scnfileitem{Все вопросы по подготовке специалистов в области \textit{Искусственного интеллекта}, а также всю научно-техническую деятельность магистрантов и аспирантов организовать через \textit{Базу знаний IMS.ostis}};
\scnfileitem{Существенно расширить библиотеки многократно используемых компонентов \textit{ostis-систем}}}
\scnnote{Фактически речь идет о поэтапном преобразовании современного представления всевозможных информационных ресурсов, различного вида стандартов, энциклопедий, википедий, толковых словарей, учебных пособий и др. в вид строгих формальных моделей, семантически совместимых друг с другом и полностью понятных, готовых к использованию интеллектуальными компьютерными системами. Одно дело -- систематизировать некоторый материал, а другое дело -- его постоянно совершенствовать и повышать эффективность его использования. Практическая ценность неразвиваемых текстов весьма низка из-за быстрого их морального старения.}


\scnheader{Стандарт OSTIS}
\scnrelfromvector{общие принципы организации эволюционных работ}{\scnfileitem{Формирование работоспособного \textit{Авторского коллектива Стандарта OSTIS}}
;\scnfileitem{Формирование \textit{Редакционной коллегии Стандарта OSTIS} для контроля целостности и качества \textit{Стандарта OSTIS} в \uline{каждый} момент времени}
;\scnfileitem{Формирование \textit{Консорциума OSTIS} для международного продвижения \textit{Стандарта OSTIS}, для взаимодействия с международными структурами, занимающимися стандартизацией \textit{интеллектуальных компьютерных систем} и \textit{технологий Искусственного интеллекта}};
\scnfileitem{Повышение качества \textit{подготовки специалистов в области Искусственного интеллекта} в вузах РБ (БГУИР, БрГТУ, БГУ, БНТУ, ГрГТУ, ПГУ) путём:
    \begin{scnitemize}
        \item тесного сотрудничества и унификации \textit{подготовки специалистов в области Искусственного интеллекта} в разных вузах;
        \item интеграции учебно-методических материалов в состав \textit{Стандарта OSTIS};
        \item непосредственного подключения студентов, магистрантов и аспирантов к реальному процессу эволюции \textit{Стандарта OSTIS}, т.е. путём непосредственного включения студентов, магистрантов и аспирантов в состав \textit{Авторского коллектива Стандарта OSTIS} со всеми вытекающими отсюда возможностями, правами и обязанностями.
    \end{scnitemize}};
\scnfileitem{Перед началом каждой (ежегодной) \textit{конференции OSTIS} осуществлять издание очередной \textit{официальной версии Стандарта OSTIS}, в которой отражаются основные изменения и дополнения \textit{Стандарта OSTIS}, внесенные в \textit{Стандарт OSTIS} за истёкший год после проведения предыдущей\textit{ конференции OSTIS}. При этом речь идет не только о содержательных (семантических) изменениях, но и об изменениях структуризации материала, изменениях в правилах и стиле оформления материала.}
;\scnfileitem{Существенно повысить уровень конструктивности и полезности каждой \textit{конференции OSTIS} для ускорения темпов \textit{эволюции стандарта OSTIS}.\\ Каждая \textit{конференция OSTIS} должна быть посвящена:
    \begin{scnitemize}
        \item подведению итогов \textit{эволюции Стандарта OSTIS} за истёкший год;
        \item анализу текущего состояния \textit{Стандарта OSTIS};
        \item уточнению наиболее актуальных направлений эволюции \textit{Стандарта OSTIS} (в первую очередь -- на следующий год).
    \end{scnitemize}}}
\begin{scnindent}
\newpage
\scnnote{Таким образом, \textit{конференции OSTIS} должны стать ежегодной площадкой для согласования и координации деятельности в направлении \textit{эволюции Стандарта OSTIS}, а также в направлении \textit{подготовки специалистов в области Искусственного интеллекта}.\\
Координация деятельности необходима
    \begin{scnitemize}
        \item не только между различными кафедрами различных \textit{вузов}, осуществляющими \textit{подготовку специалистов в области Искусственного интеллекта},
        \item но и между различными членами и группами \textit{Авторского коллектива Стандарта OSTIS},
        \item a также между \textit{Авторским коллективом Стандарта OSTIS}, \textit{Редакционной коллегией Стандарта OSTIS} и \textit{Консорциумом OSTIS}.
    \end{scnitemize}}
\end{scnindent}

\scnrelfromvector{план издания официальных версий}{Стандарт OSTIS-2021 \\
\begin{scnindent}
\scnidtf{Официальная версия \textit{Стандарта OSTIS}, публикуемая (издаваемая) до начала проведения конференции OSTIS-2021 (16-18 сентября 2021 года)}\\
\scniselement{текст, построенный на основе \textit{русскоязычной терминологии}}
\begin{scnindent}
\scnnote{При этом возможны некоторые англоязычные заимствования -- SC-код, sc-текст и др.}
\end{scnindent}
\end{scnindent}
;Стандарт OSTIS-2022
\begin{scnindent}
\scnidtf{Официальная версия \textit{Стандарта OSTIS}, публикуемая до Конференции OSTIS-2022 (апрель 2022 года)}
\scniselement{текст, построенный на основе \textit{русскоязычной терминологии}}
\end{scnindent}
;Стандарт OSTIS-2023
\begin{scnindent}
\scnidtf{\uline{Специальное} \uline{англоязычное} официальное издание версии \textit{Стандарта OSTIS}, публикуемое до \textit{Конференции OSTIS-2023} (апрель 2023 года) и ориентированное на широкий круг \uline{международной} научно-технической общественности}
\scniselement{текст, построенный на основе \textit{англоязычной терминологии}}
\scnnote{Данное англоязычное издание \textit{Стандарта OSTIS} рассматривается нами как повод, визитная карточка, приглашение к переговорам с зарубежными коллегами и организациями, занимающимися стандартизацией \textit{интеллектуальных компьютерных систем} и технологий}
\scnnote{В перспективе по мере возникновения необходимости мы будем переиздавать (в расширенном и дополнительном варианте) на английском языке некоторые версии \textit{Стандарта OSTIS} для активизации различного рода международных переговоров.}
\scnnote{При ежегодном переиздании версий \textit{Стандарта OSTIS} для "внутреннего пользования" -- для работы \textit{Редакционной коллеги Стандарта OSTIS} и \textit{Авторского коллектива Стандарта OSTIS} мы будем ориентироваться на \uline{интеграцию} использования как русскоязычной, так и англоязычной терминологии, что фактически означает включение в состав \textit{Стандарта OSTIS} русско-английского и англо-русского словарей}
\end{scnindent}
;Стандарт OSTIS-2023
\begin{scnindent}
\scnidtf{Официальная версия \textit{Стандарта OSTIS}, публикуемая до \textit{конференции OSTIS-2023} (апрель 2023 года) и использующая в равной степени как русскоязычную, так и англоязычную терминологию}
\scniselement{текст, построенный на основе интеграции русскоязычной и англоязычной терминологии}
\end{scnindent}
;Стандарт OSTIS-2024
}

\scnrelfrom{авторский коллектив}{Авторский коллектив Стандарта OSTIS}
\begin{scnindent}
\scnnote{Работоспособность, квалификация и результативность \textit{Авторского коллектива Стандарта OSTIS} определяют темпы и качество эволюции \textit{Стандарта OSTIS}}
\scnnote{При подготовке к изданию каждой официально фиксируемой версии \textit{Стандарта OSTIS}, как правило, будут формироваться специальные группы из общего \textit{Авторского коллектива Стандарта OSTIS}, каждая из которых ориентируется на подготовку к изданию соответствующей версии \textit{Стандарта OSTIS}}
\end{scnindent}
\scnheader{автор Стандарта OSTIS}
\scnidtf{соавтор Стандарта OSTIS}
\scnidtf{член Авторского коллектива Стандарта OSTIS}
\scnnote{соавтором \textit{Стандарта OSTIS} может быть любой желающий, способный согласовывать свою персональную инициативную деятельность и свою точку зрения с другими соавторами \textit{Стандарта OSTIS}}

\scnheader{Стандарт OSTIS}
\scnrelfrom{редакционная коллегия}{Редакционная коллегия Стандарта OSTIS}
\begin{scnindent}
\scnidtf{редколлегия стандарта OSTIS}
\scnidtf{Редакционная коллегия, обеспечивающая развитие \textit{Стандарта OSTIS}, совмещенного (интегрированного) с комплексным учебно-методическим обеспечением \textit{подготовки специалистов в области Искусственного интеллекта}.}
\scntext{несёт ответственность за}{Корректность (непротиворечивость), системность, целостность, полноту всех разрабатываемых материалов \textit{Стандарта OSTIS} и, в том числе, \textit{учебно-методического обеспечения подготовки специалистов в области Искусственного интеллекта}.}
\scnidtf{Рабочий орган, обеспечивающий организацию коллективного творческого процесса по развитию \textit{Стандарта OSTIS}, совмещенного с учебно-методическим обеспечением подготовки соответствующих специалистов.}
\scnidtf{Редакционная коллегия, несущая ответственность за качество перманентно эволюционируемого \textit{Стандарта OSTIS}.}
\scnidtf{Редакционная коллегия, осуществляющая координацию деятельности авторов разработки очередной (следующей) версии \textit{Стандарта OSTIS} к следующей \textit{конференции OSTIS}.}
\scnnote{В частности, \textit{Редколлегия Стандарта OSTIS} может осуществлять распределение работ по построению следующей версии \textit{Стандарта OSTIS} с четкой привязкой ответственных авторов \textit{Стандарта OSTIS} к соответствующим разделам \textit{Стандарта OSTIS}.}
\scnidtf{Очень важная структура, определяющая научно-технический уровень, авторитет и репутацию всей \textit{Технологии OSTIS} в глазах международной научно-технической общественности.}
\scnnote{Каждый член \textit{Редакционной коллегии Стандарта OSTIS} должен быть достаточно активным членом \textit{Авторского коллектива Стандарта OSTIS}. Все члены \textit{Редакционной коллегии Стандарта OSTIS} должны иметь научную степень не ниже кандидата наук.}
\end{scnindent}
\scnrelfrom{консорциум}{Консорциум OSTIS}
\begin{scnindent}
\scnrelfromlist{несёт ответственность за}{\scnfileitem{Распространение, внедрение и продвижение комплекса стандартов \textit{Технологии OSTIS} во все виды \textit{человеческой деятельности} в рамках глобального комплексного прикладного проекта \textit{Экосистемы OSTIS}.};
\scnfileitem{Взаимодействие с международными институтами и консорциумами, заинтересованными в стандартизации \textit{интеллектуальных компьютерных систем} и соответствующих технологий их \textit{проектирования}, \textit{производства}, \textit{эксплуатации} и \textit{реинжиниринга}.}}
\end{scnindent}

\newpage
\scnheader{консорциум*}
\scnidtf{быть консорциумом для заданного инновационного продукта, которым, в частности, может быть стандарт некоторые перспективные технологии*}
\scnidtf{быть субъектом продвижение заданного инновационного продукта на международной арене*}
\scniselement{бинарное ориентированное отношение}

\scnheader{Стандарт OSTIS}
\scntext{аналоги}{Аналогами (в широком смысле) \textit{Стандарта OSTIS} можно считать:
    \begin{scnitemize}
        \item любую серьезную попытку систематизации результатов, полученных в области \textit{Искусственного интеллекта} к текущему моменту:
        \begin{scnitemizeii}
            \item учебник, достаточно полно отражающий текущее состояние \textit{Искусственного интеллекта};
            \item справочник, содержащий достаточно полную информацию о текущем состоянии \textit{Искусственного интеллекта}. Примером такого справочника является трехтомный справочник по Искусственному интеллекту (\scncite{AIHandbook}, \scncite{AIHandbookMM}, \scncite{AIHandbookPAS}).
        \end{scnitemizeii}
        \item любую попытку перехода от частных формальных моделей различных компонентов интеллектуальных компьютерных систем общей (объединённой, интегрированной) формальной модели \textit{интеллектуальных компьютерных систем} в целом -- к общей теории \textit{интеллектуальных компьютерных систем};
        \item любую попытку унификации технических решений, устранения "вредного" многообразия форм технических решений при разработке \textit{интеллектуальных компьютерных систем}
        \item первые попытки разработки стандартов \textit{интеллектуальных компьютерных систем} и \textit{технологий Искусственного интеллекта}, которые чаще всего ограничиваются построением систем соответствующих понятий.
    \end{scnitemize}
}
\scntext{сравнение с аналогами}{Перечислим основные особенности и достоинства \textit{Стандарта OSTIS} по сравнению с различного вида его аналогами:
    \begin{scnitemize}
        \item \textit{Стандарт OSTIS} -- это не просто систематизация современного состояния результатов в области \textit{Искусственного интеллекта}, это систематизация, представленная в виде общей комплексной \uline{формальной} модели \textit{интеллектуальных компьютерных систем} и комплексной \uline{формальной} модели \textit{технологии} их разработки. Более того, текст \textit{Стандарта OSTIS} представляет собой раздел \textit{базы знаний} специальной \textit{интеллектуальной метасистемы}, которая ориентирована:
        \begin{scnitemizeii}
            \item на поддержку эволюция \textit{Стандарта OSTIS};
            \item на поддержку разработки \textit{интеллектуальных компьютерных систем} различного назначения;
            \item на поддержку \textit{подготовки специалистов в области Искусственного интеллекта};
        \end{scnitemizeii}
        \item \textit{Стандарт OSTIS} -- это \uline{динамический} текст, перманентно отражающий все новые и новые научно-технические результаты, получаемые в области \textit{Искусственного интеллекта} в рамках \textit{Общей теории интеллектуальных компьютерных систем} и \textit{Общей комплексной технологии разработки интеллектуальных компьютерных систем}. Здесь важной является оперативность фиксации новых научно-технических результатов, т.е. минимизация отрезка времени между моментом получения новых результатов и моментом интеграции описания этих результатов в состав \textit{Стандарта OSTIS}. В перспективе авторы новых научно-технических результатов в области \textit{Искусственного интеллекта} будут заинтересованы лично публиковать (интегрировать) свои результаты в состав \textit{Стандарта OSTIS}, т.е. становиться соавторами \textit{Стандарта OSTIS}, чтобы обеспечить необходимую оперативность такой публикации и отсутствие искажений своих результатов. Динамичность \textit{Стандарта OSTIS} и достаточная оперативность интеграции в его состав новых научно-технических результатов в области \textit{Искусственного интеллекта} делает \textit{Стандарт OSTIS} всегда актуальным и никогда морально устаревшим;
        \item В рамках стандарта OSTIS нет противопоставления между научно-технической информацией, добываемой в области Искусственного интеллекта, и учебно-методической информацией, используемой для подготовки и самоподготовки специалистов в области Искусственного интеллекта. информация о том, чему учить, должна быть "переплетена", интегрирована с информацией о том, как учить.
    \end{scnitemize}
}

\scnheader{следует отличать*}
\scnhaselementset{Стандарт OSTIS\\
\begin{scnindent}
\scnexplanation{как \uline{внутреннее} представление Стандарта OSTIS в памяти ostis-системы}
\end{scnindent}
;Технология OSTIS\\
\begin{scnindent}
\scnexplanation{как объект, специфицируемый (описываемый) стандартом OSTIS}
\end{scnindent}
;официальная версия Стандарта OSTIS\\
\begin{scnindent}
\scnhaselement{Стандарт OSTIS-2021}
\begin{scnindent}
\scnexplanation{как \uline{внешнее} представление Стандарта OSTIS в виде исходного текста соответствующего раздела базы знаний}
\end{scnindent}
\end{scnindent}}

\bigskip


\scnheader{Стандарт OSTIS-2021}
\scnrelto{официальная версия}{Стандарт OSTIS}
\scnnote{Данная официально изданная версия \textit{Стандарта OSTIS}, которую Вы держите в руках, занимает особое место:
    \begin{scnitemize}
        \item Во-первых, это первый опыт издания (публикации) подобного документа, в рамках которого необходимо обеспечивать, с одной стороны, строгую формальность, а, с другой стороны, интуитивное и адекватное понимание формальных текстов со стороны читателей;
        \item Во-вторых, данный текст является описанием условно выделенной первой версии \textit{Стандарта OSTIS} (\textit{Стандарта OSTIS-2021}), в рамках которого представлены далеко не все разделы \textit{Стандарта OSTIS}. Эти разделы будут представлены в последующих версиях \textit{Стандарта OSTIS} (в \textit{Стандарте OSTIS-2022}, в \textit{Стандарте OSTIS-2023} и т.д.);
        \item Особенностью \textit{публикации} (издания) Стандарта OSTIS версии OSTIS-2021, как, впрочем, и всех последующих версий, является то, что она оформлена в виде \uline{внешнего представления} основной части \textit{базы знаний} специальной \textit{ostis-системы}, которая предназначена для комплексной поддержки проектирования \uline{семантически совместимых} \textit{ostis-систем}. Эту систему мы назвали \textit{Метасистемой IMS.ostis} (Intelligent MetaSystem for ostis-systems). Последовательность изложения материала во внешнем представлении \textit{базы знаний} не является единственно возможным маршрутом прочтения (просмотра) \textit{базы знаний}. Каждый читатель, войдя в \textit{Метасистему IMS.ostis}, может выбрать любой другой маршрут навигации по этой \textit{базе знаний}, задавая указанной метасистеме те \textit{вопросы}, которые в текущий момент его интересуют. Таким образом, читая предлагаемый вашему вниманию текст и одновременно работая с \textit{Метасистемой IMS.ostis}, можно значительно быстрее усвоить детали \textit{Технологии OSTIS} и значительно быстрее приступить к непосредственному использованию указанной технологии. Этому также способствует большое количество примеров семантических моделей различных фрагментов \textit{интеллектуальных компьютерных систем};
        \item Основной семантический вид \textit{разделов баз знаний ostis-систем} -- это формальное представление различных \textit{предметных областей} вместе с соответствующими им \textit{онтологиями}. При этом явно указываются связи между этими \textit{предметными областями} и \textit{онтологиями}. Таким образом, \textit{база знаний} \scnbigspace \textit{Метасистемы IMS.ostis}, как и любых других \textit{интеллектуальных компьютерных систем}, построенных по \textit{Технологии OSTIS}, представляет собой иерархическую систему связанных между собой формальных моделей \textit{предметных областей} и соответствующих им \textit{онтологий}. Соответственно этому структурирован и текст \textit{Стандарта OSTIS-2021};
        \item В основе \textit{Технологии OSTIS} лежит предлагаемая нами унификация \textit{интеллектуальных компьютерных систем}, основанная, в свою очередь, на \textit{смысловом представлении знаний} в \textit{памяти интеллектуальных компьютерных систем}. Таким образом, данную \textit{публикацию} \textit{Стандарта OSTIS-2021} можно рассматривать как версию \textit{стандарта} семантических моделей \textit{интеллектуальных компьютерных систем}. Последующие \textit{публикации}, посвящённые детальному описанию различных компонентов \textit{Технологии OSTIS}, будут также оформляться как внешнее представление соответствующих \textit{разделов базы знаний} \scnbigspace \textit{Метасистемы IMS.ostis} и будут отражать следующие этапы развития  \textit{Технологии OSTIS}, следующие версии этой технологии, и, соответственно, следующие версии \textit{Метасистемы IMS.ostis};
        \item Все основные положения \textit{Технологии OSTIS} рассматривались и обсуждались на ежегодных \textit{конференциях OSTIS}, которые стали важным стимулирующим фактором становления и развития \textit{Технологии OSTIS}. Мы благодарим всех активных участников этих конференций;
        \item Важной задачей \textit{Стандарта OSTIS-2021} была выработка стилистики формализованного представления научно-технической информации, которая одновременно была бы понятна как человеку, так и интеллектуальной компьютерной системе. По сути это принципиально новый подход к оформлению научно-технических результатов, позволяющий:
        \begin{scnitemizeii}
            \item существенно повысить уровень автоматизации анализа качества (корректности, целостности) научно-технической информации;
            \item интеллектуальным компьютерным системам непосредственно (без какой-либо дополнительной "ручной" доработки) использовать информацию (знания), содержащуюся в разработанных специалистами документах;
            \item существенно упростить согласование точек зрения различных специалистов, входящих в коллектив разработчиков той или  иной научно-технической документации.
        \end{scnitemizeii}
        \item Для выработки стилистики и формального представления научно-технической информации нам было важно привлечь к обсуждению и анализу материала \textit{Стандарта OSTIS-2021} как можно больше коллег, участвующих в развитии и применении \textit{Технологии OSTIS}. При этом некоторых коллег мы включили в число соавторов соответствующих разделов монографии. Основной целью написания \textit{Стандарта OSTIS-2021} является создание технологических и организационных предпосылок к принципиально новому  подходу к организации \textit{научно-технической деятельности} в любой области и, в частности, в области создания и перманентного развития комплексной технологии проектирования и производства семантически совместимых интеллектуальных компьютерных систем (\textit{Технологии OSTIS}). Суть указанного подхода заключается  в глубокой конвергенции и интеграции результатов деятельности всех специалистов, участвующих в создании и развитии \textit{Технологии OSTIS}, путем организации коллективной разработки \textit{базы знаний}, являющейся формальным представлением полной \textit{Документации Технологии OSTIS}, отражающей текущее состояние этой технологии.
    \end{scnitemize}
}
\scntext{соавторы}{На данном этапе к разработке и оформлению различных разделов текущей версии \textit{Стандарта OSTIS-2021} кроме основных авторов были привлечены студенты, магистранты, аспиранты и преподаватели кафедры Интеллектуальных информационных технологий Белорусского государственного университета информатики и радиоэлектроники и кафедры Интеллектуальных информационных технологий Брестского государственного технического университета, а также сотрудники ОАО «Савушкин продукт» и ООО «Интелиджент семантик системс». Так, например,
    \begin{scnitemize}
        \item соавторами Раздела \scnqqi{\nameref{sec:sd_neuronetworks}} являются Головко В.А., Ковалёв М.В., Крощенко А.А., Михно Е.В.;
        \item соавтором Разделов \scnqqi{\nameref{sd_sem_ui}}, \scnqqi{\nameref{sec:sd_interfaces}}, \scnqqi{\nameref{sd_user_interface_actions}} является Садовский М.Е.;
        \item соавторами Раздела \scnqqi{\nameref{sec:sd_natural_languages}} являются Гордей А.Н., Никифоров С.А., Бобёр Е.С., Святощик М.И.;
        \item соавторами Сегмента \scnqqi{\textit{Предметная область и онтология субъектно-объектных спецификаций воздействий}} в Разделе \scnqqi{\nameref{sec:sd_actions}} являются Гордей А.Н., Никифоров С.А., Бобёр Е.С., Святощик М.И.;
        \item соавтором Раздела \scnqqi{\nameref{sd_program_interp}} является Корончик Д.Н.;
        \item соавторами Раздела \scnqqi{\nameref{sec:sd_ecosys_enterprise}} являются Таберко В.В., Иванюк Д.С., Касьяник В.В., Пупена А.Н.
    \end{scnitemize}
}
\scntext{благодарности}{Благодарим студентов Банцевич К.А., Бутрина С.В., Василевскую А.П., Менькову Е.А., Жмырко А.В., Григорьеву И.В., Загорского А.Г., Марковца В.С., Киневича Т.О. и сотрудников кафедры Интеллектуальных информационных технологий Белорусского государственного университета информатики и радиоэлектроники за оказание технической помощи при подготовке текста к печати, сотрудников кафедры Интеллектуальных информационных технологий Брестского государственного технического университета, сотрудников ОАО <<Савушкин продукт>>, а также выражаем благодарность ООО <<Интелиджент семантик системс>> и его генеральному директору Т. Грюневальду за финансовую поддержку работ по развитию \textit{Технологии OSTIS}, а также финансовую поддержку издания \textit{Стандарта OSTIS}.}
\scnidtf{Издание Документации Технологии OSTIS-2021}
\scnidtf{Первое издание (публикация) Внешнего представления Документации Технологии OSTIS в виде книги}
\scniselement{публикация}
\begin{scnindent}
\scnidtf{библиографический источник}
\end{scnindent}
\scniselement{официальная версия Стандарта OSTIS}
\scniselement{бумажное издание}
\scniselement{научное издание}
\scnrelfrom{рекомендация издания}{Совет БГУИР}
\scnrelfromset{рецензенты}{Курбацкий А.Н.; Дудкин А.А.}
\scnrelfrom{издательство}{Бестпринт}
\begin{scnindent}
\scntext{УДК}{004.8}
\end{scnindent}
\scntext{ISBN}{978-985-7267-13-2}
